This is an implementation of a ray tracer from \href{https://raytracing.github.io/books/RayTracingInOneWeekend.html#metal}{\tt {\itshape Ray Tracing in One Weekend}} as well as \href{https://devblogs.nvidia.com/accelerated-ray-tracing-cuda/}{\tt {\itshape Accelerated Ray Tracing in One Weekend in C\+U\+DA}}

The two program creates a simple image of various size spheres.

 {\itshape Image Produced with Cuda}

 $\ast$\+Image Produced with C++$\ast$

The purpose of writing the two programs that output a similiar image is to show to the performance increase that using C\+U\+DA provides.

 {\itshape Profiling of Cuda program}

 {\itshape Profiling of C++ program}

As shown by the profiling, using Cuda\textquotesingle{}s parallel processing platform allows for almost 400 times faster rendering of the image.

\subsection*{\#\# Working with the C++ Ray Tracer }

 These instructions will get you a copy of the project up and running on your local machine for development and testing purposes.

\subsubsection*{Prerequisites}

To run this code you need\+:
\begin{DoxyEnumerate}
\item C++ Compiler(I used g++)
\item P\+PM viewer(I used gimp, also online viewer here \href{http://paulcuth.me.uk/netpbm-viewer/}{\tt http\+://paulcuth.\+me.\+uk/netpbm-\/viewer/})
\end{DoxyEnumerate}

\subsubsection*{Installation}


\begin{DoxyEnumerate}
\item Download all files in the v1 folder
\item Compile and run tracer.\+cpp with your C++ compiler
\item Should produce a final2.\+ppm image
\end{DoxyEnumerate}

\subsection*{\#\# Modifications }

To create your own scene, you have to create a list of sphere objects. 
\begin{DoxyCode}
hittable *list[5];
list[0] = new sphere(vec3(0,0,-1), 0.5, new lambertian(vec3(0.1, 0.2, 0.5)));
list[1] = new sphere(vec3(0,-100.5,-1), 100, new lambertian(vec3(0.8, 0.8, 0.0)));
list[2] = new sphere(vec3(1,0,-1), 0.5, new metal(vec3(0.8, 0.6, 0.2), 0.3));
list[3] = new sphere(vec3(-1,0,-1), 0.5, new dielectric(1.5));
list[4] = new sphere(vec3(-1,0,-1), -0.45, new dielectric(1.5));
hittable *world = new hittable\_list(list,5);
\end{DoxyCode}
 This is a example of list of 5 spheres Notice to create a sphere object it takes 3 parameter
\begin{DoxyEnumerate}
\item Vector
\item Radius
\item Material
\end{DoxyEnumerate}

The vector defines the center location of the sphere, taking x,y,z coordinates {\ttfamily vec3(x,y,z)}

The radius is a float the set the radius of the sphere

The material sets how the sphere should look, there are 3 types of materials
\begin{DoxyEnumerate}
\item Metals, which are reflective
\item Lambertian, which are not reflective
\item Dielectrics which act like glass 
\end{DoxyEnumerate}